\section{Additive Manufacturing}
\label{sec:additive-manufacturing}

Textkörper mit Formel:

\begin{equation}
    U(j\omega)=\int^{\infty}_{-\infty}{u(t) \cdot e^{-j\omega t}dt}
    \label{form:form1}
\end{equation}

Textkörper Fortsetzung mit Verweis auf Formel \ref{form:form1}. Und nicht zu vergessen: es gibt auch noch eine tolle Abbildung in Kapitel \ref{ch:intro}, nämlich Abbildung \ref{fig:buecher}.

\subsection{Selective Laser Sintering - SLS}
\label{subsec:selective-laser-sintering---sls}

Textkörper mit direktem Zitat und Seitenanzahl:
``It would be very easy to show how technical or report writing differed from other writing'' \cite[p.~3]{young2002technical}.

\subsection{Problems of Additive Manufacturing}
\label{subsec:problems-of-additive-manufacturing}

Textkörper mit Referenzen:
Für weiterführende Informationen zum wissenschaftlichen Schreiben siehe "J. Schimel, Writing Science" \cite{schimel2012writing}. Es wird empfohlen den Sprachleitfaden der FH Campus Wien \cite{alker2006} zu berücksichtigen und die Checkliste für wissenschafltiches Schreiben \cite{petz2018} zu verwenden. Beide Leitfäden sind im FH Portal zu finden.

\subsection{Possible Solutions}
\label{subsec:possible-solutions}
